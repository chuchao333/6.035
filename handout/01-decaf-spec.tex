\documentstyle[11pt,bnf,6035]{article}

{\bnf
\gdef\kw#1{{\bnf K[#1]}}
\gdef\term#1{{\bnf T[#1]}}
\gdef\nt#1{{\bnf <#1>}}}

\begin{document}
\handout{Handout --- Decaf Language}{\CLASSONE}

The project for the course is to write a compiler for a language
called Decaf.  Decaf is a simple imperative language similar to C or
Pascal.

\section*{Lexical Considerations}

All Decaf keywords are lowercase.  Keywords and
identifiers are case-sensitive.  For example, \kw{if} is a keyword,
but {\tt IF} is a variable name; {\tt foo} and {\tt Foo} are two
different names referring to two distinct variables.

The reserved words are:


\kw{boolean break callout class continue else false for
  if int return true void}

Note that \kw{Program} (see below) is not a keyword, but an identifier
with a special meaning in certain circumstances.

Comments are started by {\tt //} and are terminated by the end of the
line.

White space may appear between any lexical tokens.  White space is
defined as one or more spaces, tabs, line-break
characters, and comments.

Keywords and identifiers must be separated by white space, or a token
that is neither a keyword nor an identifier.  For example, {\tt
thisfortrue} is a single identifier, not three distinct keywords.
If a sequence begins with an alphabetic character or an underscore,
then it, and the longest sequence of characters following it forms a
token.

String literals are composed of {\bnf <char>}s enclosed in double
quotes.  A character literal consists of a {\bnf <char>} enclosed in
single quotes.

Numbers in Decaf are 32 bit signed. That is, decimal values between
-2147483648 and 2147483647.  If a sequence begins with {\tt 0x}, then
these first two characters and the longest sequence of characters
drawn from {\tt [0-9a-fA-F]} form a hexadecimal integer literal.  If a
sequence begins with a decimal digit (but not {\tt 0x}), then the
longest prefix of decimal digits forms a decimal integer literal.
Note that range checking is performed later.  A long sequence of
digits (e.g. 123456789123456789) is still scanned as a single token.

A {\bnf <char>} is any printable ASCII character (ASCII values between
decimal value 32 and 126, or octal 40 and 176) other than quote
(\verb+"+), single quote (\verb+'+), or backslash (\verb+\+), plus the
2-character sequences ``\verb+\"+'' to denote quote, ``\verb+\'+'' to
denote single quote, ``\verb+\\+'' to denote backslash, ``\verb+\t+'' to
denote a literal tab, or ``\verb+\n+'' to denote newline.

\newpage
\section*{Reference Grammar}

Meta-notation:

{\def\x{{\rm $x$}}
\def\y{{\rm $y$}}
\begin{center}
\begin{tabular}{|c|l|} \hline
\nt{foo}&               means foo is a nonterminal. \\
\kw{foo}&               (in \kw{bold} font) means that \kw{foo} is a
                        terminal \\
&           i.e., a token or a part of a token.\\
{\bnf O[T[\x]]}&        means zero or one occurrence of \x,
                        \ie\ \x\ is optional; \\
&           \hspace*{2em} note that brackets in quotes \term{$'$\[$'$ $'$\]$'$} are terminals.\\
{\bnf R[\x]}&        means zero or more occurrences of \x. \\
{\bnf L[\x][T[,]]}&  a comma-separated list of one or more x's.\\
{\bnf G[\ \ ]}&     large braces are used for grouping;\\
&           \hspace*{2em} note that braces in quotes \term{$'$\{$'$ $'$\}$'$} are terminals.\\
{\bnf !}&               separates alternatives. \\ \hline
\end{tabular}
\end{center}}

\begin{bnfgrammar}
<program> -> K[class] K[Program] T[~$'$\{$'$~] R[<field_decl>] R[<method_decl>] T[~$'$\}$'$]

<field_decl> -> <type> L[G[<id> ! <id> T[~$'$\[$'$] <int_literal> T[$'$\]$'$~] ]][T[,]] T[~~;]

<method_decl> -> G[<type> ! K[void]] <id> T[~(~] O[L[G[<type> <id>]][T[,]]] T[~)~] <block>

<block> -> T[$'$\{$'$~] R[<var_decl>]  R[<statement>] T[~$'$\}$'$]

<var_decl> -> <type> L[<id>][T[,]] T[~~;]

<type> -> K[int] ! K[boolean]

<statement> -> <location> <assign_op> <expr> T[~~;]
             | <method_call> T[~~;]
             | K[if] T[~(~] <expr> T[~)~] <block> O[K[else] <block>]
             | K[for] <id> T[~=~] <expr> T[~,~] <expr> <block>
             | K[return] O[<expr>] T[~~;]
             | K[break] T[~~;]
             | K[continue] T[~~;]
             | <block>

<assign_op> -> T[=]
             | T[+=]
	     | T[-=]	

<method_call> -> <method_name> T[~(~] O[L[<expr>][T[,]]] T[~)~]
        | K[callout] T[~(~] <string_literal> O[T[,] L[ <callout_arg>][T[,]]] T[~)~]

<method_name> -> <id>

<location> -> <id>
            | <id> T[~$'$\[$'$~] <expr> T[~$'$\]$'$~]

\end{bnfgrammar}

\begin{bnfgrammar}
<expr> -> <location>
        | <method_call>
        | <literal>
        | <expr> <bin_op> <expr>
        | T[-] <expr>
    | T[!] <expr>
        | T[(~] <expr> T[~)]

<callout_arg> -> <expr> ! <string_literal>

<bin_op> -> <arith_op> ! <rel_op> ! <eq_op> ! <cond_op>

<arith_op> -> T[+] ! T[-] ! T[*] ! T[/] ! T[\%] % ! T[<<] ! T[>>]

<rel_op> -> T[<] ! T[>] ! T[<=] ! T[>=]

<eq_op>   -> T[==] ! T[!=]

<cond_op> -> T[\&\&] ! T[||]

<literal> -> <int_literal> ! <char_literal> ! <bool_literal>

<id> -> <alpha> R[<alpha_num>]

<alpha_num> -> <alpha> ! <digit>

<alpha> -> T[a] ! T[b] ! T[\ldots{}] ! T[z] ! T[A] ! T[B] ! T[\ldots{}] ! T[Z] ! T[\_]

<digit> -> T[0] ! T[1] ! T[2] ! T[\ldots] ! T[9]

<hex_digit> -> <digit> ! T[a] ! T[b] ! T[c] ! T[d] ! T[e] ! T[f] ! T[A] ! T[B] ! T[C] ! T[D] ! T[E] ! T[F]

<int_literal> -> <decimal_literal> ! <hex_literal>

<decimal_literal> -> <digit> R[<digit>]

<hex_literal> -> T[0x] <hex_digit> R[<hex_digit>]

<bool_literal> -> K[true] ! K[false]

<char_literal> -> T['] <char> T[']

<string_literal> -> T["] R[<char>] T["]
\end{bnfgrammar}

\section*{Semantics}
\def\arrayint{\kw{array}\term{\[\kw{int}\]}}
\def\arraybool{\kw{array}\term{\[\kw{bool}\]}}
\def\arrayT{\kw{array}\term{\[T\]}}

A Decaf program consists of a single class declaration
for a class called \kw{Program}. The class declaration consists of
field declarations and method declarations.  Field declarations
introduce variables that can be accessed globally by all methods in
the program.  Method declarations introduce functions/procedures.  The
program must contain a declaration for a method called \kw{main} that
has no parameters.  Execution of a Decaf program starts
at method \kw{main}.

\subsection*{Types}

There are two basic types in Decaf --- \kw{int} and
\kw{boolean}.  In addition, there are arrays of integers (\term{int \[
$N$ \]}) and arrays of booleans (\term{boolean \[ $N$ \]}).

Arrays may be declared only in the global (class declaration) scope.
All arrays are one-dimensional and have a compile-time fixed size.
Arrays are indexed from 0 to $N-1$, where $N > 0$ is the size of the
array.  The usual bracket notation is used to index arrays.  Since
arrays have a compile-time fixed size and cannot be declared as
parameters (or local variables), there is no facility for querying the
length of an array variable in Decaf.

\subsection*{Scope Rules}

Decaf has simple and quite restrictive scope rules.  All
identifiers must be defined (textually) before use.  For example:

 \begin{itemize}
  \item a variable must be declared before it is used.
  \item a method can be called only by code appearing after its header.
(Note that recursive methods are allowed.)
 \end{itemize}

There are at least two valid scopes at any point in a
Decaf program: the global scope, and the method
scope.  The global scope consists of names of fields and methods
introduced in the (single) \kw{Program} class declaration.  The
method scope consists of names of variables and formal parameters
introduced in a method declaration.  Additional local scopes exist within
each \nt{block} in the code; these can come after \kw{if} or \kw{for}
statements, or inserted anywhere a \nt{statement} is
legal.  An identifier introduced in a method scope can shadow an
identifier from the global scope.  Similarly, identifiers introduced
in local scopes shadow identifiers in less deeply nested scopes, the
method scope, and the global scope.

Variable names defined in the method scope or a local scope may
shadow method names in the global scope.  In this case, the
identifier may only be used as a variable until the variable leaves
scope.

No identifier may be defined more than once in the same scope.  Thus
field and method names must all be distinct in the global scope, and
local variable names and formal parameters names must be distinct in
each local scope.

\subsection*{Locations}

Decaf has two kinds of locations: local/global scalar
variables and (global) array elements.  Each location has a type.
Locations of types \kw{int} and \kw{boolean} contain integer values and
boolean values, respectively.  Locations of types \term{int \[ $N$ \]}
and \term{boolean \[ $N$ \]} denote array elements.  Since arrays are
statically sized in Decaf, arrays may be allocated in the
static data space of a program and need not be allocated on the heap.

Each location is initialized to a default value when it is declared.
Integers have a default value of zero, and booleans have a default
value of \kw{false}.  Local variables must be initialized when the
declaring scope is entered.  Array elements are initialized when the
program starts.

\subsection*{Assignment}

Assignment is only permitted for scalar values.  For the types
\kw{int} and \kw{boolean}, Decaf uses value-copy semantics, and the
assignment {\bnf <location>~T[=]~<expr>} copies the value resulting
from the evaluation of {\bnf <expr>} into {\bnf <location>}.  The
{\bnf <location>~T[+=]~<expr>} assignment increments the value
stored in {\bnf <location>} by {\bnf <expr>}, and is only valid for
both {\bnf <location>} and {\bnf <expr>} of type \kw{int}. The
{\bnf <location>~T[-=]~<expr>} assignment decrements the value
stored in {\bnf <location>} by {\bnf <expr>}, and is only valid for
both {\bnf <location>} and {\bnf <expr>} of type \kw{int}.

The {\bnf <location>} and the {\bnf <expr>} in an assignment must
have the same type. For array types, {\bnf <location>} and {\bnf
<expr>} must refer to a single array element which is also a scalar
value.

It is legal to assign to a formal parameter variable within a method
body.  Such assignments affect only the method scope.

\subsection*{Method Invocation and Return}

Method invocation involves (1) passing argument values from the caller
to the callee, (2) executing the body of the callee, and (3) returning
to the caller, possibly with a result.

Argument passing is defined in terms of assignment: the formal arguments
of a method are considered to be like local variables of the method and
are initialized, by assignment, to the values resulting from the
evaluation of the argument expressions.  The arguments are evaluated
from left to right.

The body of the callee is then executed by executing the statements of
its method body in sequence.

A method that has no declared result type can only be called as a
statement, {\em i.e.}, it cannot be used in an expression.  Such a
method returns control to the caller when \kw{return} is called (no
result expression is allowed) or when the textual end of the callee is
reached.

A method that returns a result may be called as part of an expression,
in which case the result of the call is the result of evaluating the
expression in the \kw{return} statement when this statement is reached.
It is illegal for control to reach the textual end of a method that
returns a result.

A method that returns a result may also be called as a statement.  In
this case, the result is ignored.

\subsection*{Control Statements}

\subsubsection*{\kw{if}}
The \kw{if} statement has the usual semantics.  First, the {\bnf <expr>}
is evaluated.  If the result is \kw{true}, the \kw{true} arm is
executed.  Otherwise, the \kw{else} arm is executed, if it exists.
Since Decaf requires that the \kw{true} and \kw{else} arms
be enclosed in braces, there is no ambiguity in matching an \kw{else}
arm with its corresponding \kw{if} statement.

% The \kw{while} statement has the usual semantics.  First, the {\bnf
% <expr>} is evaluated.  If the result is \kw{false}, control exits the
% loop.  Otherwise, the loop body is executed.  If control reaches the end
% of the loop body, the \kw{while} statement is executed again.

\subsubsection*{\kw{for}}
The \kw{for} statement is similar to a {\kw do} loop in Fortran.  The
{\bnf <id>} is the loop index variable and it shadows a variable of
the same name declared in an outer scope if one exists.  The loop
index variable declares an integer variable whose scope is limited to
the body of the loop.   The first {\bnf<expr>} is the initial value
of the loop index variable and the second {\bnf<expr>} is the ending
value of the loop index variable.  Each of these expressions are
evaluated once, just prior to reaching the loop for the first time.
Each expression must evaluate to an integer value.
The loop body is executed if the current value of the index variable
is less than the ending value.  After an execution of the
loop body, the index variable in incremented by 1, and the new value
is compared to the ending value to decide if another iteration should
execute.

\subsection*{Expressions}

Expressions follow the normal rules for evaluation.  In the absence of
other constraints, operators with the same precedence are evaluated from
left to right.  Parentheses may be used to override normal precedence.

A location expression evaluates to the value contained by the location.

Method invocation expressions are discussed in {\em Method Invocation
and Return}.  Array operations are discussed in {\em Types}.  I/O
related expressions are discussed in {\em Library Callouts}.

Integer literals evaluate to their integer value.  Character literals
evaluate to their integer ASCII values, \eg\ \term{'A'} represents the
integer 65.  (The type of a character literal is \kw{int}.)

The arithmetic operators ({\bnf <arith_op>} and unary minus) have
their usual precedence and meaning, as do the relational operators
({\bnf <rel_op>}).  {\tt \%} computes the remainder of dividing its
operands. 

%The {\tt <<} is a left shift and the {\tt >>} is an
%arithmetic right shift.

Relational operators are used to compare integer expressions.  The
equality operators, \term{==} and \term{!=} are defined for \kw{int}
and \kw{boolean} types only, can be used to compare any two
expressions having the same type.  (\term{==} is ``equal'' and
\term{!=} is ``not equal'').

The result of a relational operator or equality operator has type
\kw{boolean}.

The boolean connectives \term{\&\&} and \term{||} are interpreted using
short circuit evaluation as in Java.  The side-effects of the second
operand are not executed if the result of the first operand determines
the value of the whole expression (i.e., if the result is false for
\term{\&\&} or true for \term{||}).

Operator precedence, from highest to lowest:

\begin{center}
\begin{tabular}{|c|l|}
\hline
{\em Operators\/}  & {\em Comments\/} \\
\hline
\term{-}              & unary minus \\
\term{!}              & logical not \\
\term{* / \%}         & multiplication, division, remainder \\
\term{+ -}            & addition, subtraction \\
%\term{<< >>}          & shifts\\
\term{<  <=   >=  >}  & relational \\
\term{==  !=}         &  equality \\
\term{\&\&}           & conditional and \\
\term{||}             & conditional or \\ \hline
\end{tabular}
\end{center}

Note that this precedence is not reflected in the reference grammar.

\subsection*{Library Callouts}

Decaf includes a primitive method for calling functions
provided in the runtime system, such as the standard C library or
user-defined functions.

The primitive method for calling functions is:

\begin{quotation}
\noindent \kw{int} \kw{callout} ({\bnf <string_literal>,
O[L[<callout_arg>][T[,]]]} ) --- the function named by the initial
string literal is called and the arguments supplied are passed to
it. Expressions of boolean or integer type are passed as integers;
string literals or expressions with array type are passed as
pointers. The return value of the function is passed back as an
integer. The user of \kw{callout} is responsible for ensuring that the
arguments given match the signature of the function, and that the return
value is only used if the underlying library function actually returns a
value of appropriate type. Arguments are passed to the function in the
system's standard calling convention.
\end{quotation}

In addition to accessing the standard C library using \kw{callout}, an
I/O function can be written in C (or any other language), compiled using
standard tools, linked with the runtime system, and accessed by the
\kw{callout} mechanism.


% rules-decaf.tex -- shared between decaf spec and the semantic assignment.

\subsection*{Semantic Rules}

These rules place additional constraints on the set of valid
Decaf programs besides the constraints implied by the
grammar.  A program that is grammatically well-formed and does not
violate any of the following rules is called a \textit{legal} program.
A robust compiler will explicitly check each of these rules, and will
generate an error message describing each violation it is able to
find.  A robust compiler will generate at least one error message for
each illegal program, but will generate no errors for a legal program.

\begin{enumerate}
\item No identifier is declared twice in the same scope.
\item No identifier is used before it is declared.
\item The program contains a definition for a method called 
      \kw{main} that has no parameters (note that since execution starts at
       method \kw{main}, any methods defined after main will never be 
      executed).
\item The {\bnf <int_literal>} in an array declaration must be greater
      than 0.
\item The number and types of arguments in a method call (non-callout) must be the same
      as the number and types of the formals, i.e., the signatures
must be identical.
\item If a method call is used as an expression, the method must
      return a result.
\item String literals may not be used as arguments to non-callout methods.
\item A \kw{return} statement must not have a return value unless it appears
      in the body of a method that is declared to return a value.
\item The expression in a \kw{return} statement must have the same type as
      the declared result type of  the enclosing method definition.
%\item If a method returns a non-void type, all exits from the method must
%      explicitly return an expression of return type.
\item An {\bnf <id>} used as a {\bnf <location>} must name a declared 
      local/global variable or formal parameter.
\item For all locations of the form {\bnf <id>T[\[]<expr>T[\]]}
      \begin{enumerate}
      \item {\bnf <id>} must be an 
            \kw{array} variable, and 
      \item the type of {\bnf <expr>} must be \kw{int}.
      \end{enumerate}
\item The {\bnf <expr>} in an \kw{if} statement must have type
      \kw{boolean}.
\item The operands of {\bnf <arith_op>}s and {\bnf <rel_op>}s must have type
      \kw{int}.
\item The operands of {\bnf <eq_op>}s must have the same type, either
      \kw{int} or \kw{boolean}.
\item The operands of {\bnf <cond_op>}s and the operand of logical not
      (\kw{!}) must have type \kw{boolean}.
\item The {\bnf <location>} and the {\bnf <expr>} in an assignment, 
{\bnf <location>~T[=]~<expr>}, must have the same type.
\item The {\bnf <location>} and the {\bnf <expr>} in an
incrementing/decrementing assignment,
{\bnf <location>~T[+=]~<expr>} and {\bnf <location>~T[-=]~<expr>}, must be of type \kw{int}.
\item The initial {\bnf <expr>} and the ending {\bnf <expr>} of
  \kw{for} must have type \kw{int}.
\item All \kw{break} and \kw{continue} statements must be contained within 
      the body of a \kw{for}.
\end{enumerate}


\section*{Run Time Checking}

In addition to the constraints described above, which are statically
enforced by the compiler's semantic checker, the following constraints
are enforced dynamically: the compiler's code generator must emit
code to perform these checks; violations are discovered at run-time.

\begin{enumerate}
\item The subscript of an array must be in bounds.
\item Control must not fall off the end of a method that is declared to
      return a result.
\end{enumerate}

When a run-time error occurs, an appropriate error message is output to
the terminal and the program terminates.  Such error messages should be
helpful to the programmer trying to find the problem in the source
program.

\end{document}
