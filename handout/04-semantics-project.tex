\documentstyle[11pt,bnf,6035]{article}
\begin{document}
\thispagestyle{empty}

{\bnf
\gdef\kw#1{{\bnf K[#1]}}
\gdef\term#1{{\bnf T[#1]}}
\gdef\nt#1{{\bnf <#1>}}}

\handout{Handout --- Semantic Analysis Project}{\CHECKERASSIGN}

{\bf DUE: \CHECKERDUE}

Extend your compiler to find, report, and recover from semantic
errors in Decaf programs. Most semantic errors can be checked by
testing the rules enumerated in the section ``Semantic Rules'' of
Decaf Spec. These rules are repeated in this handout as well.
However, you should read Decaf Spec in its entirety to make sure that
your compiler catches all semantic errors implied by the language
definition. We have attempted to provide a precise statement of the
semantic rules. If you feel some semantic rule in the language
definition may have multiple interpretations, you should work with
the interpretation that sounds most reasonable to you and clearly
list your assumptions in your project documentation (you can also
send email to {\tt 6.035-staff@mit.edu} asking for clarification).

This part of the project includes the following tasks:

\begin{enumerate}

\item Create a high-level intermediate representation (IR) tree. You can do
this either by instructing ANTLR to create one for you, adding
actions to your grammar to build a tree, or walking a generic ANTLR
tree and building a tree on the way. The problem of designing an IR
will be discussed in the lectures; some hints are given in the final
section of this handout.

When running in debug mode, your driver should pretty-print the
constructed IR tree in some easily-readable form suitable for
debugging.

\item Build symbol tables for the classes.  (A symbol table is an
environment, i.e. a mapping from identifiers to semantic objects such
as variable declarations.  Environments are structured hierarchically,
in parallel with source-level constructs, such as class-bodies,
method-bodies, loop-bodies, etc.)

\item Perform all semantic checks by traversing the IR and
accessing the symbol tables. {\bf Note:} the run-time checks are
not required for this assignment.

\end{enumerate}


\subsection*{What to Hand In}

Follow the directions given in project overview handout when writing up
your project.  Your design documentation should include a description of how
your IR and symbol table structures are organized, as well as a discussion
of your design for performing the semantic checks.

Each group will be assigned an athena group and shared space in the course locker.
Each group must place their completed submission at:\[
\mbox{/mit/6.035/group/{\bf GROUP}/submit/{\bf GROUP}-semantics.tar.gz}
\]
Groups may use:\[
\mbox{/mit/6.035/group/{\bf GROUP}/}
\]
as a place to collaborate and share code.  It is recommended that you use
revision control and place your repositories in this shared space.

Submitted tarballs should have the following structure:
{\scriptsize
\begin{verbatim}
GROUPNAME-semantics.tar.gz
|
`-- GROUPNAME-semantics
    |
    |-- AUTHORS            (list of students in your group, one per line)
    |
    |-- code
    |   |
    |   ...                (full source code, can build by running `ant`)
    |
    |-- doc 
    |   |
    |   ...                (write-up, described in project overview handout)
    |   
    `-- dist 
        |
        `-- Compiler.jar   (compiled output, for automated testing)
\end{verbatim}
}

You should be able to run
your compiler from the command line with:
\begin{verbatim}
  java -jar dist/Compiler.jar -target inter <filename>
\end{verbatim}
The resulting output to the terminal should be a report of all
errors encountered while compiling the file.  Your compiler should
give reasonable and specific error messages (with line and column
numbers and identifier names) for all errors detected. It should
avoid reporting multiple error messages for the same error. For
example, if {\tt y} has not been declared in the assignment
statement ``{\tt x=y+1;}'', the compiler should report only one
error message for {\tt y}, rather than one error message for {\tt
y}, another error message for the {\tt +}, and yet another error
message for the assignment.

After you implement the static semantic checker, your compiler should
be able to detect and report {\em all} static (i.e., compile-time)
errors in any input Decaf program, including lexical and syntax errors
detected by previous phases of the compiler. In addition, your
compiler should not report any messages for valid Decaf
programs. However, we do not expect you to avoid reporting spurious
error messages that get triggered by error recovery.  It is possible
that your compiler might mistakenly report spurious semantic errors in
some cases depending on the effectiveness of your parser's syntactic
error recovery.

As mentioned, your compiler should have a debug mode in which the IR
and symbol table data structures constructed by your compiler are
printed in some form.  This can be run from the command line by
\begin{verbatim}
  java -jar dist/Compiler.jar -target inter -debug <filename>
\end{verbatim}

\subsection*{Test Cases}

The test cases provided for this project are in:
\begin{verbatim}
  /mit/6.035/provided/semantics/
\end{verbatim}

Read the comments in the test cases to see what we expect your
compiler to do.  Points will be awarded based on how well your
compiler performs on these and hidden tests cases.  Complete
documentation is also required for full credit.

\subsection*{Grading Script}

As with the previous project, we are providing you with the grading script
we will use to test your code (except for the write-up).  This script only
shows you results for the public test cases.

The script can be found on athena in the course locker: 
\[
\mbox{/mit/6.035/provided/gradingscripts/p2grader.py}
\]

The script takes your tarball as the only arg:
\[
\mbox{/mit/6.035/provided/gradingscripts/p2grader.py GROUPNAME-parser.tar.gz}
\]

Or it works on the extracted directory:
\[
\mbox{/mit/6.035/provided/gradingscripts/p2grader.py GROUPNAME-parser/}
\]

Please test your submission against this script.  It may help you debug your
code and it will help make the grading process go more smoothly.


% rules-decaf.tex -- shared between decaf spec and the semantic assignment.

\subsection*{Semantic Rules}

These rules place additional constraints on the set of valid
Decaf programs besides the constraints implied by the
grammar.  A program that is grammatically well-formed and does not
violate any of the following rules is called a \textit{legal} program.
A robust compiler will explicitly check each of these rules, and will
generate an error message describing each violation it is able to
find.  A robust compiler will generate at least one error message for
each illegal program, but will generate no errors for a legal program.

\begin{enumerate}
\item No identifier is declared twice in the same scope.
\item No identifier is used before it is declared.
\item The program contains a definition for a method called 
      \kw{main} that has no parameters (note that since execution starts at
       method \kw{main}, any methods defined after main will never be 
      executed).
\item The {\bnf <int_literal>} in an array declaration must be greater
      than 0.
\item The number and types of arguments in a method call (non-callout) must be the same
      as the number and types of the formals, i.e., the signatures
must be identical.
\item If a method call is used as an expression, the method must
      return a result.
\item String literals may not be used as arguments to non-callout methods.
\item A \kw{return} statement must not have a return value unless it appears
      in the body of a method that is declared to return a value.
\item The expression in a \kw{return} statement must have the same type as
      the declared result type of  the enclosing method definition.
%\item If a method returns a non-void type, all exits from the method must
%      explicitly return an expression of return type.
\item An {\bnf <id>} used as a {\bnf <location>} must name a declared 
      local/global variable or formal parameter.
\item For all locations of the form {\bnf <id>T[\[]<expr>T[\]]}
      \begin{enumerate}
      \item {\bnf <id>} must be an 
            \kw{array} variable, and 
      \item the type of {\bnf <expr>} must be \kw{int}.
      \end{enumerate}
\item The {\bnf <expr>} in an \kw{if} statement must have type
      \kw{boolean}.
\item The operands of {\bnf <arith_op>}s and {\bnf <rel_op>}s must have type
      \kw{int}.
\item The operands of {\bnf <eq_op>}s must have the same type, either
      \kw{int} or \kw{boolean}.
\item The operands of {\bnf <cond_op>}s and the operand of logical not
      (\kw{!}) must have type \kw{boolean}.
\item The {\bnf <location>} and the {\bnf <expr>} in an assignment, 
{\bnf <location>~T[=]~<expr>}, must have the same type.
\item The {\bnf <location>} and the {\bnf <expr>} in an
incrementing/decrementing assignment,
{\bnf <location>~T[+=]~<expr>} and {\bnf <location>~T[-=]~<expr>}, must be of type \kw{int}.
\item The initial {\bnf <expr>} and the ending {\bnf <expr>} of
  \kw{for} must have type \kw{int}.
\item All \kw{break} and \kw{continue} statements must be contained within 
      the body of a \kw{for}.
\end{enumerate}


\section*{Implementation Suggestions}

\begin{itemize}

\item
You will need to declare classes for each of the nodes in your IR.  In
many places, the hierarchy of IR node classes will resemble the
language grammar.  For example, a part of your inheritance tree might
look like this (where indentation represents inheritance):


\begin{small}\begin{verbatim}
abstract class Ir
abstract class     IrExpression
abstract class         IrLiteral
         class             IrIntLiteral
         class             IrBooleanLiteral
         class         IrCallExpr
         class             IrMethodCallExpr
         class             IrCalloutExpr
         class         IrBinopExpr
abstract class     IrStatement
         class         IrAssignStmt
         class         IrPlusAssignStmt
         class         IrBreakStmt
         class         IrContinueStmt
         class         IrIfStmt
         class         IrForStmt
         class         IrReturnStmt
         class         IrInvokeStmt
         class         IrBlock
         class     IrClassDecl
abstract class     IrMemberDecl
         class         IrMethodDecl
         class         IrFieldDecl
         .
         .
         .
         class     IrVarDecl
         class     IrType
\end{verbatim}\end{small}

Classes such as these implement the {\it abstract syntax tree} of
the input program.  In its simplest form, each class is simply a tuple
of its subtrees, for example:

\begin{small}\begin{verbatim}
public class IrBinopExpr extends IrExpression
{
    private final int          operator;                      |
    private final IrExpression lhs;                           +
    private final IrExpression rhs;                          / \
}                                                         lhs   rhs
\end{verbatim}\end{small}

\noindent or:

\begin{small}\begin{verbatim}
public class IrAssignStmt extends IrStatement
{                                                             :=
    private final IrLocation   lhs;                          /  \
    private final IrExpression rhs;                       lhs    rhs
}
\end{verbatim}\end{small}

In addition, you'll need to define classes for the semantic entities
of the program, which represent abstract properties (e.g. expression
types, method signatures, class descriptors, etc.) and to establish
the correspondences between them.  Some examples: every expression has
a type; every variable declaration introduces a variable; every block
defines a scope.  Many of these properties are derived by recursive
traversals over the tree.

As far as possible, you should try to make instances of the the symbol
classes {\it canonical}, so that they may be compared using reference
equality.  You are strongly advised to design these classes with care,
employing good software-engineering practice and documenting them, as
you will be living with them for the next few months!

\item
All error messages should be accompanied by the filename, line and
column number of the token most relevant to the error message (use
your judgement here).  This means that, when building your
abstract-syntax tree (or AST), you must ensure that each IR node
contains sufficient information for you to determine its line number
at some later time.

% (Some industrial compilers, e.g. the Jikes compiler for Java,
% associates with each IR node the {\it start} and {\it end} character
% positions of the first and last tokens included in that construct,
% allowing its error messages to report exactly which part of a source
% line contains an error.  This can also be done easily with the JavaCUP
% framework, which exposes the left and right character positions of
% each grammar symbol {\it n} in the variables {\tt left} and {\tt
% right}.)

It is {\it not} appropriate to throw an exception when encountering an
error in the input: doing so would lead to a design in which at most
one error message is reported for each run of the compiler.  A good
front-end saves the user time by reporting multiple errors before
stopping, allowing the programmer to make several corrections before
having to restart compilation.

\item
Semantic checking should be done {\bf top-down}. While the
type-checking component of semantic checking can be done in
bottom-up fashion, other kinds of checks (for example, detecting
uses of undefined variables) can not.

There are two ways of achieving this. The first is to make use of
parser actions {\it in the middle of productions}. This approach may
require less code but can be more complex, because more work needs
to be done directly within the ANTLR infrastructure.

A cleaner approach is to invoke your semantic checker on a complete
AST after parsing has finished.  The pseudocode for {\tt block} in
this approach would resemble this:

\begin{small}\begin{verbatim}
   void checkBlock(EnvStack envs, Block b) {
       envs.push(new Env());
       foreach s in b.statements
           checkStatement(envs, s);
       envs.pop();
   }
\end{verbatim}\end{small}

In this pseudocode, a new environment is created and pushed on the
environment stack, the body of the block is then checked in the
context of this environment stack, and then the new environment is
discarded when the end of the block is reached.

The semantic check can thus be expressed as a {\it visitor} (which
should be familiar from {\it Design Patterns} or 6.170) over the
AST.  (Since code generation, certain optimizations, and
pretty-printing can also be naturally expressed as visitors, we
recommend this approach for a cleaner implementation.)

\item
The treatment of negative integer literals requires some care. Recall
from the previous handout that negative integer literals are in fact
two separate tokens: the positive integer literal and a preceding
`{\tt -}'.  Whenever your top-down semantic checker finds a unary
minus operator, it should check if its operand is a positive integer
literal, and if so, replace the subtree (both nodes) by a single,
negative integer literal.

\end{itemize}
\end{document}
