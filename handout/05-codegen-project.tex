\documentstyle[11pt,bnf,6035]{article}
\begin{document}

\handout{Handout --- Code Generation Project}{\CODEGENASSIGN}

{\bf Checkpoint and Design Document DUE: \CODEGENDESIGNDUE}

{\bf DUE: \CODEGENDUE}

Code Generation involves producing correct x86-64 assembler code for
all Decaf programs.  The two following projects will involve code
optimizations.  For now, we are not interested in whether your
generated code is efficient.

By the end of code generation, you should have a fully working Decaf
compiler.  You'll be able to write, compile, and execute real
programs on a real machine!

\section*{Project Assignment}

For Code Generation, your compiler will translate your high-level IR
into a low-level IR.  Your low-level IR will include structures that
more closely match the machine instructions of a modern
architecture.  Your compiler will then translate your low-level IR
into x86-64 assembly code to be run on an AMD Opteron machine. You
should target the subset of the x86-64 ISA defined in the
\textit{x86-64 Architecture Guide}, which is an appendix to this
document. For a given input file containing a Decaf program, your
compiler must generate an assembly language listing
($\left<filename\right>$\/{\tt .s}\/).

Your generated code must include all runtime checks listed in
language specification.  Additional checks such as integer overflow are not
required.

The two final assignments, {\em Dataflow Analysis} and {\em
Optimization}, will focus on improving the efficiency of the target
code generated by your compiler. For this assignment, you are not
expected to produce great code.  (Even horrendous code is
acceptable.  When considering tradeoffs, always choose simplicity of
implementation over performance.)

You are not constrained as to how you go about generating your final
assembly code listing.  However, we suggest that you follow the
general approach presented in lecture.

You will have a number of opportunities to do some creative design
work for the code optimization projects.  For this assignment, you
should focus your creative energies on designing your low-level IR,
familiarizing yourself with our target ISA, your machine-code
representations of the run-time structures, and generating correct
assembly code.  Do not try to produce an improved register allocation
scheme; you will be addressing these issues later.

\section*{System Usage}

We have setup two machines for this course for you to use. These
are:
\begin{itemize}
\item {\tt tyner.csail.mit.edu}
\item {\tt silver.csail.mit.edu}
\end{itemize}
These machines are setup to be part of the CAG (Compilers and
Architecture) group at CSAIL. The accounts and passwords are
different from your regular Athena accounts. Each group will use a
single account in the form {\tt le0X}. Your passwords have been put
in your Athena lockers in files {\tt le0X-pass}. Please change your
passwords immediately. You are free to use these machines for your
6.035 related needs. You may, however, find it more convenient to do
most of the development on a laptop or desktop, and use these
servers for testing. This would be particularly true if you use an
IDE like Eclipse for your development.

\section*{Compiling and Libraries}

Your compiler will create a {\tt .s} file, which can be compiled to
create an executable file. You can use {\tt gcc} to compile
your assembly code.

Decaf does not have any input/output functions. Part of the
assignment is to implement the standard x86-64 calling convention
for {\tt callout} statements, so that you can interface with the
outside world. Any function that is called using {\tt callout} needs
to be linked in separately. {\tt gcc} will link against any standard
libraries (you may need to use the {\tt -l} argument for {\tt gcc}
to link some libraries). The testing files provided to you link
against the standard C library. If you want to use functions that
are not easy to use in Decaf (handle pointers, etc), you are welcome
to write your own library calls in {\tt C}, compile them to object
files (using {\tt gcc -c}) and then link them in by hand when
compiling your assembly.

\section*{What to Hand In}

Follow the directions given in project overview handout when writing up
your project.  Your design documentation should include a description of
how your IR and code generation are organized, as well as a discussion of
your design for generating code.

Each group must place their completed submission at:\[
\mbox{/mit/6.035/group/{\bf GROUP}/submit/{\bf GROUP}-codegen.tar.gz}
\]

Submitted tarballs should have the following structure:
{\scriptsize
\begin{verbatim}
GROUPNAME-codegen.tar.gz
|
`-- GROUPNAME-codegen
    |
    |-- AUTHORS            (list of students in your group, one per line)
    |
    |-- code
    |   |
    |   ...                (full source code, can build by running `ant`)
    |
    |-- doc 
    |   |
    |   ...                (write-up, described in project overview handout)
    |   
    `-- dist 
        |
        `-- Compiler.jar   (compiled output, for automated testing)
\end{verbatim}
}

You should be able to run your compiler from the command line with:
\begin{verbatim}
  java -jar dist/Compiler.jar -target codegen <filename>
\end{verbatim}
Your compiler should then write a x86-64 assembly listing to: {\tt <filename>.s}

Nothing should be written to standard out or standard error for a
syntactically and semantically correct program unless the {\tt -debug}
flag is present.  If the {\tt -debug} flag is present, your compiler
should still run and produce the same resulting assembly listing.  Any
debugging output is left to your own discretion.

\subsection*{Grading Script}

As with the previous project, we are providing you with the grading script
we will use to test your code (except for the write-up).  This script only
shows you results for the public test cases.

The script can be found on athena in the course locker: 
\[
\mbox{/mit/6.035/provided/gradingscripts/p3grader.py}
\]

The script takes your tarball as the only arg:
\[
\mbox{/mit/6.035/provided/gradingscripts/p3grader.py GROUPNAME-codegen.tar.gz}
\]

Or it works on the extracted directory:
\[
\mbox{/mit/6.035/provided/gradingscripts/p3grader.py GROUPNAME-codegen/}
\]

Please test your submission against this script.  It may help you debug your
code and it will help make the grading process go more smoothly.


\subsection*{Checkpoint and Design Document}
For this stage of the project, you are required to provide a
checkpoint of your implementation and to submit a design document one
week prior to the due date of the project.  Follow the submission
instructions above for the checkpoint.  The TA will grab the
checkpoint files from your group directory at 11:59 on the checkpoint
due date.  Your compiler does not have to produce correct code at the
time of the checkpoint. The checkpoint exists to encourage you to
start working early on the project. If you get your project working at
the end, the checkpoint will have little effect. However, if your
group is unable to complete the project, the checkpoint submission
will play a critical role in your grade. If we determine that your
group did not do a substantial amount of work before the checkpoint,
you will be severely penalized.


The design document should describe your design and implementation
plans for the code generation stage.  This description will include
the design of your low-level IR and a discussion of your code
generation scheme and implementation.  This document also counts
towards the project grade.


\subsection*{Test Cases}

We will run your compilers on the test cases in:
\begin{verbatim}
  /mit/6.035/provided/codegen/
\end{verbatim}
and on a set of hidden tests.

\subsection*{Related Handouts}

The \textit{X86-64 Architecture Guide}, provided as a supplement to
this handout, presents a not-so-gentle introduction to the x86-64 ISA.  It
presents a subset of the x86-64 architecture that should be sufficient
for the purposes of this project. You should read this handout before
you start to write the code that traverses your IR and generates
x86-64 instructions.

\end{document}
